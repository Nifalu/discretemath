
%% General definitions
\documentclass{article} %% Determines the general format.
\usepackage{a4wide} %% paper size: A4.
\usepackage[utf8]{inputenc} %% This file is written in UTF-8.
%% Some editors on Windows cannot save files in UTF-8.
%% If there is a problem with special characters not showing up
%% correctly, try switching "utf8" to "latin1" (ISO 8859-1).
\usepackage[T1]{fontenc} %% Format of hte resulting PDF file.
\usepackage{fancyhdr} %% Package to create a header on each page.
\usepackage{lastpage} %% Used for "Page X of Y" in the header.
											%% For this to work, you have to call pdflatex twice.
\usepackage{enumerate} %% Used to change the style of enumerations (see below).

\usepackage{amssymb} %% Definitions for math symbols.
\usepackage{amsmath} %% Definitions for math symbols.
\usepackage{amsthm}
\usepackage{braket}
\usepackage{graphicx}
\usepackage{float}
\usepackage{hyperref}

\usepackage{tikz}  %% Pagacke to create graphics (graphs, automata, etc.)
\usetikzlibrary{automata} %% Tikz library to draw automata
\usetikzlibrary{arrows}   %% Tikz library for nicer arrow heads


%% Left side of header
\lhead{\course\\\semester\\Exercise \homeworkNumber}
%% Right side of header
\rhead{\authorname\\Page \thepage\ of \pageref{LastPage}}
%% Height of Header
\usepackage[headheight=36pt]{geometry}
%% Page style that uses the header
\pagestyle{fancy}

\newcommand{\authorname}{Nico Bachmann\\Ruben Hutter\\Lina Mehrle}
\newcommand{\semester}{Fall Semester 2023}
\newcommand{\course}{Discrete Mathematics in Computer Science}
\newcommand{\homeworkNumber}{6}


\begin{document}

\section*{Exercise \homeworkNumber.1}

$dom(f) = \mathbb{N}$\newline
$img(f) = \set{n \in \mathbb{N} \mid n \equiv 0 \text{ (mod 3)}}$


\section*{Exercise \homeworkNumber.2}

\begin{equation}
f(p) = \begin{cases}
		v & \text{if } p = 0,1\\
		x & \text{if } p = 4\\
		y & \text{if } p = 2\\
		z & \text{if } p = 5\\
	\end{cases}
\end{equation}

\section*{Exercise \homeworkNumber.3}

Let $A = \set{0}$, $B = \set{0, 1}$, $C = \set{0}$

\begin{equation}
f(0) = 0
\end{equation}
\begin{equation}
g(0) = g(1) = 0
\end{equation}
\noindent
$g$ is not injective because both $0$ and $1$ point to $0$. However $g \circ f$ is still injective because $1$ is not in the image of $f$.



\section*{Exercise \homeworkNumber.4}

\begin{proof}
We will show that f is bijective by separately proving that f is injective and surjective. \newline 
Step 1: Show that f is injective by contradiction. \newline
Let $x,y \in A$ such that $x \neq y$ and $f(x) = f(y)$. But then $x = g(f(x)) = g(f(y)) = y$ since by assumption $g(f(z)) = z$ for all $z \in A$. This is a contradiction to $x \neq y$. \newline
Step 2: Show that f is surjective by contradiction. \newline
Assume there is an $y \in B$ such that $y \notin img(f)$. Since $h: B \rightarrow A$ is a (total) function, there is an element $x \in A$ such that $h(y) = x$. But this means that $y = f(h(y)) = f(x)$ which is a contradiction to $y \notin img(f)$.

\end{proof}

\section*{Exercise \homeworkNumber.5}

\begin{enumerate}[a)]

\item $ P \circ C = \set{A \mapsto 4, B \mapsto 3, C \mapsto 2, D \mapsto 1, E \mapsto 5}$

\item $ R_1 = \set{1 \mapsto 1, 2 \mapsto 3, 3 \mapsto 4, 4 \mapsto 5, 5 \mapsto 2} $

\item \verb|EC|



\end{enumerate}





\end{document}
