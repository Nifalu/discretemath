%% General definitions
\documentclass{article} %% Determines the general format.
\usepackage{a4wide} %% paper size: A4.
\usepackage[utf8]{inputenc} %% This file is written in UTF-8.
%% Some editors on Windows cannot save files in UTF-8.
%% If there is a problem with special characters not showing up
%% correctly, try switching "utf8" to "latin1" (ISO 8859-1).
\usepackage[T1]{fontenc} %% Format of hte resulting PDF file.
\usepackage{fancyhdr} %% Package to create a header on each page.
\usepackage{lastpage} %% Used for "Page X of Y" in the header.
											%% For this to work, you have to call pdflatex twice.
\usepackage{enumerate} %% Used to change the style of enumerations (see below).

\usepackage{amssymb} %% Definitions for math symbols.
\usepackage{amsmath} %% Definitions for math symbols.
\usepackage{amsthm}
\usepackage{braket}
\usepackage{graphicx}
\usepackage{float}
\usepackage{hyperref}

\usepackage{tikz}  %% Pagacke to create graphics (graphs, automata, etc.)
\usetikzlibrary{automata} %% Tikz library to draw automata
\usetikzlibrary{arrows}   %% Tikz library for nicer arrow heads


%% Left side of header
\lhead{\course\\\semester\\Exercise \homeworkNumber}
%% Right side of header
\rhead{\authorname\\Page \thepage\ of \pageref{LastPage}}
%% Height of header
\usepackage[headheight=36pt]{geometry}
%% Page style that uses the header
\pagestyle{fancy}

\newcommand{\authorname}{Nico Bachmann\\Ruben Hutter\\Lina Mehrle}
\newcommand{\semester}{Fall Semester 2023}
\newcommand{\course}{Discrete Mathematics in Computer Science}
\newcommand{\homeworkNumber}{1}


\begin{document}

\section*{Exercise \homeworkNumber.1}

\begin{proof}
Let $A$,$B$ and $C$ be arbitrary sets. Let $x$ be an element in $(A \cap B) \cup (A \cap C)$. By definition of the union, either $x \in (A \cap B)$ or $ x \in (A \cap C)$. If $x \in (A \cap B)$, then $x$ is an element in both $A$ and $B$ by definition of the intersection. Analogously, if $x \in (A \cap C)$, then $x$ is an element in both $A$ and $C$ by definition of the intersection. So $x \in A$ and $x \in (B \cup C)$, i.e. $x \in A \cap (B \cup C) $, again by the definitions of the intersection and the union, which ends the proof.

\end{proof}

\section*{Exercise \homeworkNumber.2}

\begin{proof}
Let $A$ and $B$ be arbitrary sets. We want to show that if $(A \cap B) = \emptyset$, then $A \setminus B = A$. We argue by contradiction, so we assume that $(A \setminus B) \neq A$.
By definition $A \setminus B \subseteq A$, so $A 	\not\subseteq A \setminus B$. To fulfill this definition, there must be a an $x \in A \cup B$. 


\end{proof}


\section*{Exercise \homeworkNumber.3}

\begin{proof}
Let $A$ and $B$ be arbitrary sets. We want to show that if $A \cup B = B$, then $A \subseteq B$ by contrapositive. So we assume that if $A \not\subseteq B$, then $A \cup B \neq B$.\\
Since $A \not\subseteq B$, there must be an $x$ which is in $A$ but not in $B$. Therefore $A \cup B$ cannot be $B$.


\end{proof}

\section*{Exercise \homeworkNumber.4}

\begin{proof}
Let $A = \{"Lina", "Leda", "Alessia", "Scarlet"\}$, $B = \{"Lina", "Leda", "Alessia"\}$ and \linebreak $C  = \{"Leda", "Alessia", "Scarlet"\}$. It´s easy to see, that while $A$ is exactly the Set $B \cup C$, $A$ is neither a subset of $B$ or $C$.

 
\end{proof}

\end{document}
