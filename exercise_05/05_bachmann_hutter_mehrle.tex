
%% General definitions
\documentclass{article} %% Determines the general format.
\usepackage{a4wide} %% paper size: A4.
\usepackage[utf8]{inputenc} %% This file is written in UTF-8.
%% Some editors on Windows cannot save files in UTF-8.
%% If there is a problem with special characters not showing up
%% correctly, try switching "utf8" to "latin1" (ISO 8859-1).
\usepackage[T1]{fontenc} %% Format of hte resulting PDF file.
\usepackage{fancyhdr} %% Package to create a header on each page.
\usepackage{lastpage} %% Used for "Page X of Y" in the header.
											%% For this to work, you have to call pdflatex twice.
\usepackage{enumerate} %% Used to change the style of enumerations (see below).

\usepackage{amssymb} %% Definitions for math symbols.
\usepackage{amsmath} %% Definitions for math symbols.
\usepackage{amsthm}
\usepackage{braket}
\usepackage{graphicx}
\usepackage{float}
\usepackage{hyperref}

\usepackage{tikz}  %% Pagacke to create graphics (graphs, automata, etc.)
\usetikzlibrary{automata} %% Tikz library to draw automata
\usetikzlibrary{arrows}   %% Tikz library for nicer arrow heads


%% Left side of header
\lhead{\course\\\semester\\Exercise \homeworkNumber}
%% Right side of header
\rhead{\authorname\\Page \thepage\ of \pageref{LastPage}}
%% Height of Header
\usepackage[headheight=36pt]{geometry}
%% Page style that uses the header
\pagestyle{fancy}

\newcommand{\authorname}{Nico Bachmann\\Ruben Hutter\\Lina Mehrle}
\newcommand{\semester}{Fall Semester 2023}
\newcommand{\course}{Discrete Mathematics in Computer Science}
\newcommand{\homeworkNumber}{5}


\begin{document}

\section*{Exercise \homeworkNumber.1}

\begin{enumerate}[a)]

	\item There are 5 equivalence relations over the set $M = \set{a,b,c}$.
	
	\begin{enumerate}[]
		\item $\sim = \set{\langle a, a \rangle, \langle b, b \rangle, \langle c, c \rangle}$
		\item $\sim = \set{\langle a, a \rangle, \langle b, b \rangle, \langle c, c \rangle, \langle a, b \rangle, \langle b, a \rangle}$
		\item $\sim = \set{\langle a, a \rangle, \langle b, b \rangle, \langle c, c \rangle, \langle a, c \rangle, \langle c, a \rangle}$
		\item $\sim = \set{\langle a, a \rangle, \langle b, b \rangle, \langle c, c \rangle, \langle c, b \rangle, \langle b, c \rangle}$
		\item $\sim = \set{\langle a, a \rangle, \langle b, b \rangle, \langle c, c \rangle, \langle a, b \rangle, \langle b, a \rangle, \langle a, c \rangle, \langle c, a \rangle, \langle c, b \rangle, \langle b, c \rangle}$
		
	\end{enumerate}
	
	\item $[a]_\sim = \set{a}$, $[b]_\sim = \set{b,d}$, $[c]_\sim = \set{c,e}$
	
\end{enumerate}

\section*{Exercise \homeworkNumber.2}

\begin{enumerate}[a)]

\item \begin{proof}

Let $x$ be an arbitrary element in $S$. Since an equivalence relation is reflexive, we have $\langle x,x \rangle \in \sim$ which means that $x \in [x]_\sim \subseteq E$.

\end{proof}

\item \begin{proof}

We will do a proof by contradiction. Let $x,y,z$ be arbitrary elements in $S$ and assume that $x \in [y]_\sim$ and $x \in [z]_\sim$ but $[y]_\sim \neq [z]_\sim$.

Since $x \in [y]_\sim$ we have that $\langle y,x \rangle \in \sim$.
Since $x \in [z]_\sim$ we have that $\langle z,x \rangle \in \sim$ and since an equivalence relation is symmetric we also have that $\langle x,z \rangle \in \sim$.

Because an equivalence relation is also transitive, we have $\langle y,z \rangle \in \sim$. \newline
We now claim that $[y]_\sim = [z]_\sim$ if $\langle y,z \rangle \in \sim$. This would be the needed contradiction to finish the proof. \newline
Proof of the claim: Let $a \in [y]_\sim$ be arbitrary. Then $\langle a, y \rangle \in \sim$ and thus $\langle a,z \rangle \in \sim$ since $\langle y,z \rangle \in \sim$ i.e. $a \in [z]_\sim$. Thus $[y]_\sim \subseteq [z]_\sim$. Analogously $[z]_\sim \subseteq [y]_\sim$. But this means that $[y]_\sim = [z]_\sim$.

\end{proof}
\end{enumerate}


\section*{Exercise \homeworkNumber.3}

$R = \set{\langle a,a \rangle, \langle b,b \rangle, \langle c,c \rangle, \langle d,d \rangle, \langle a,b \rangle, \langle c,b \rangle}$
\newline
\newline
\noindent
Our relation $R$ is a partial order because it is reflexive, antisymmetric and transitive. It is reflexive because $\langle x,x \rangle \in R$ for all $x \in S$, antisymmetric because $\langle a,b \rangle \in R$ but $\langle b,a \rangle
\notin R$, similarly for $\langle c,b \rangle$ and transitive because $\langle a,b \rangle \in R$ but $b$ does not relate something else except itself and similarly for $\langle c,b \rangle$.
\newline
\newline
\noindent
Our minimal elements are $a$ and $c$.
\newpage

\section*{Exercise \homeworkNumber.4}

\begin{enumerate}[a)]
\item False. It should be [...] \emph{at least} one of xRy and yRx (so possible also both) is true [...].
\item You can always take the relation $R = \emptyset$. This way all $x \in S$ stand in no relation to each other and are therefore each a maximum and minimum.
\end{enumerate}

\section*{Exercise \homeworkNumber.5}

\begin{enumerate}[a)]

\item $A^{-1} = \set{\langle 2x, x \rangle \mid x \in \mathbb{N}_0}$

\item $B \setminus A^{-1} = \set{\langle i \cdot x, x \rangle \mid i,x \in \mathbb{N}_0, i \neq 2}$

\item $C \circ A = \set{\langle 2,2 \rangle, \langle 2,7 \rangle, \langle 3,4 \rangle , \langle 4,2 \rangle}$

\item $A \circ (A \circ A) = \set{\langle x, 8x \rangle \mid x \in \mathbb{N}_0}$

\item $A^* = \set{\langle x, 2^{n} \cdot x \rangle \mid x,n \in \mathbb{N}_0}$

\item $A \circ B = \set{\langle i \cdot x, 2x \rangle \mid i,x \in \mathbb{N}_0}$

\end{enumerate}





\end{document}
