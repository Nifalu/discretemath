%% General definitions
\documentclass{article} %% Determines the general format.
\usepackage{a4wide} %% paper size: A4.
\usepackage[utf8]{inputenc} %% This file is written in UTF-8.
%% Some editors on Windows cannot save files in UTF-8.
%% If there is a problem with special characters not showing up
%% correctly, try switching "utf8" to "latin1" (ISO 8859-1).
\usepackage[T1]{fontenc} %% Format of hte resulting PDF file.
\usepackage{fancyhdr} %% Package to create a header on each page.
\usepackage{lastpage} %% Used for "Page X of Y" in the header.
											%% For this to work, you have to call pdflatex twice.
\usepackage{enumerate} %% Used to change the style of enumerations (see below).

\usepackage{amssymb} %% Definitions for math symbols.
\usepackage{amsmath} %% Definitions for math symbols.
\usepackage{amsthm}
\usepackage{braket}
\usepackage{graphicx}
\usepackage{float}
\usepackage{hyperref}

\usepackage{tikz}  %% Pagacke to create graphics (graphs, automata, etc.)
\usetikzlibrary{automata} %% Tikz library to draw automata
\usetikzlibrary{arrows}   %% Tikz library for nicer arrow heads


%% Left side of header
\lhead{\course\\\semester\\Exercise \homeworkNumber}
%% Right side of header
\rhead{\authorname\\Page \thepage\ of \pageref{LastPage}}
%% Height of Header
\usepackage[headheight=36pt]{geometry}
%% Page style that uses the header
\pagestyle{fancy}

\newcommand\Circle[1][1.5]{\tikz[baseline=-#1]{\draw[thick](0,0.1)circle[radius=#1mm];}}

\newcommand{\authorname}{Nico Bachmann\\Ruben Hutter\\Lina Mehrle}
\newcommand{\semester}{Fall Semester 2023}
\newcommand{\course}{Discrete Mathematics in Computer Science}
\newcommand{\homeworkNumber}{2}


\begin{document}

\section*{Exercise \homeworkNumber.1}

\begin{proof}
We use a mathematical induction over n. \newline
Induction basis $n = 0$: 
\[ \sum_{i=0}^{0} i = 0 = \frac{0 \cdot(0+1)}{2} \]
IH: \[ \sum_{i=0}^{n} i = \frac{n \cdot (n+1)}{2}\]
Induction step $n \rightarrow n+1$:
\begin{align*}
\sum_{i=0}^{n+1} i & = \left( \sum_{i=0}^{n} i \right) + (n+1) \\
& \stackrel{IH}{=} \frac{n \cdot (n+1)}{2} + (n+1) \\
& = \frac{n \cdot (n+1)}{2} + \frac{2\cdot (n+1)}{2} \\
& = \frac{(n+1) \cdot (n+2)}{2}
\end{align*}
\end{proof}

\section*{Exercise \homeworkNumber.2}

\begin{proof}
Induction basis: \[ edges(\square) = 0 = 2 \cdot 1 - 2 = 2 \cdot leaves(\square) - 2 \]
IH: to prove that the statement is true for a composite tree $\langle L, \Circle ,R  \rangle$, we may use that it is true for the subtrees $L$ and $R$.\newline
Inductive step for $B = \langle L, \Circle ,R \rangle$:
\begin{align*}
edges(B) & = edges(\langle L, \Circle ,R \rangle) \\
& = edges(L) + edges(R) + 2 \\
& \stackrel{IH}{=} 2 \cdot leaves(L) - 2 + 2 \cdot leaves(R) - 2 + 2 \\
& = 2 \cdot (leaves(L) + leaves(R)) - 2 \\
& = 2 \cdot leaves(\langle L, \Circle ,R \rangle) - 2
\end{align*}
\end{proof}


\section*{Exercise \homeworkNumber.3}

\begin{proof}

\end{proof}

\section*{Exercise \homeworkNumber.4}

\begin{proof}
 
\end{proof}

\section*{Excercise \homeworkNumber.5}

\end{document}
