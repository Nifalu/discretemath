%% General definitions
\documentclass{article} %% Determines the general format.
\usepackage{a4wide} %% paper size: A4.
\usepackage[utf8]{inputenc} %% This file is written in UTF-8.
%% Some editors on Windows cannot save files in UTF-8.
%% If there is a problem with special characters not showing up
%% correctly, try switching "utf8" to "latin1" (ISO 8859-1).
\usepackage[T1]{fontenc} %% Format of hte resulting PDF file.
\usepackage{fancyhdr} %% Package to create a header on each page.
\usepackage{lastpage} %% Used for "Page X of Y" in the header.
											%% For this to work, you have to call pdflatex twice.
\usepackage{enumerate} %% Used to change the style of enumerations (see below).

\usepackage{amssymb} %% Definitions for math symbols.
\usepackage{amsmath} %% Definitions for math symbols.
\usepackage{amsthm}
\usepackage{braket}
\usepackage{graphicx}
\usepackage{float}
\usepackage{hyperref}

\usepackage{tikz}  %% Pagacke to create graphics (graphs, automata, etc.)
\usetikzlibrary{automata} %% Tikz library to draw automata
\usetikzlibrary{arrows}   %% Tikz library for nicer arrow heads


%% Left side of header
\lhead{\course\\\semester\\Exercise \homeworkNumber}
%% Right side of header
\rhead{\authorname\\Page \thepage\ of \pageref{LastPage}}
%% Height of Header
\usepackage[headheight=36pt]{geometry}
%% Page style that uses the header
\pagestyle{fancy}

\newcommand{\authorname}{Nico Bachmann\\Ruben Hutter\\Lina Mehrle}
\newcommand{\semester}{Fall Semester 2023}
\newcommand{\course}{Discrete Mathematics in Computer Science}
\newcommand{\homeworkNumber}{4}


\begin{document}

\section*{Exercise \homeworkNumber.1}

$A = \set{1,2,3,4}$, $B = \set{3,4,5}$

\section*{Exercise \homeworkNumber.2}

\begin{enumerate}[a)]
	\item 0100 1011
	\item $\set{e_1, e_2, e_3, e_5, e_6}$
	\item 
	\begin{enumerate}[]
		\item Intersection $\rightarrow$ AND
		\item Union $\rightarrow$ OR
		\item Negation $\rightarrow$ NOT
	\end{enumerate}
\end{enumerate}


\section*{Exercise \homeworkNumber.3}

We assume $A = \mathbb{N}_0$ and $B = \mathbb{Z}_- = \set{-1, -2, ...}$. Then $|A| = |\mathbb{N}_0|$ and $|A \cup B| = |\mathbb{Z}| = |\mathbb{N}_0|$ by Exercise 3.4. So $|A| = |A \cup B|$.

\section*{Exercise \homeworkNumber.4}

$0 \mapsto 0$\\
$1 \mapsto 1$\\
$2 \mapsto -1$\\
$3 \mapsto 2$\\
$4 \mapsto -2$\\
\vdots\\\\
In the image we only have elements from $\mathbb{Z}$ and we map to every element of $\mathbb{Z}$ just once.

\section*{Exercise \homeworkNumber.5}

\begin{proof}
We can use the same Proof idea as in the slides for the proof that $\mathbb{Q}_+$ is countably infinite and just prepend a minus sign in front of every fraction in order to proof that $\mathbb{Q}_-$ is also countably infinite. By the theorem proven in the lecture we know that the union of $\mathbb{Q}_-$, $\mathbb{Q}_+$ and $\set{0}$ is countable.
\end{proof}

\newpage

\section*{Exercise \homeworkNumber.6}
TBS: The set of tarradiddles $T$ is countable.\\
Remark: We will use 1, 2, 3 instead of the Taradiddle symbols.
\begin{proof}
We define a first set containing just the elementary symbols: $\Sigma_1 = \set{1, 2, 3}$. Then we define a set containing all possible combinations of element tuples: $\Sigma_2 = \set{11, 12, 13, 21, 22, 23, 31, 32, 33}$, analogously also the sets $\Sigma_3$, $\Sigma_4$ and so forth. Those sets are countable since they are finite. The set $\bigcup_{i = 1}^\infty \Sigma_i$ is countable by the lectures theorem and so is the subset $T$ of tarradiddles.
\end{proof}

\end{document}
